% Options for packages loaded elsewhere
\PassOptionsToPackage{unicode}{hyperref}
\PassOptionsToPackage{hyphens}{url}
%
\documentclass[
]{article}
\usepackage{amsmath,amssymb}
\usepackage{iftex}
\ifPDFTeX
  \usepackage[T1]{fontenc}
  \usepackage[utf8]{inputenc}
  \usepackage{textcomp} % provide euro and other symbols
\else % if luatex or xetex
  \usepackage{unicode-math} % this also loads fontspec
  \defaultfontfeatures{Scale=MatchLowercase}
  \defaultfontfeatures[\rmfamily]{Ligatures=TeX,Scale=1}
\fi
\usepackage{lmodern}
\ifPDFTeX\else
  % xetex/luatex font selection
\fi
% Use upquote if available, for straight quotes in verbatim environments
\IfFileExists{upquote.sty}{\usepackage{upquote}}{}
\IfFileExists{microtype.sty}{% use microtype if available
  \usepackage[]{microtype}
  \UseMicrotypeSet[protrusion]{basicmath} % disable protrusion for tt fonts
}{}
\makeatletter
\@ifundefined{KOMAClassName}{% if non-KOMA class
  \IfFileExists{parskip.sty}{%
    \usepackage{parskip}
  }{% else
    \setlength{\parindent}{0pt}
    \setlength{\parskip}{6pt plus 2pt minus 1pt}}
}{% if KOMA class
  \KOMAoptions{parskip=half}}
\makeatother
\usepackage{xcolor}
\usepackage[margin=2cm]{geometry}
\usepackage{graphicx}
\makeatletter
\def\maxwidth{\ifdim\Gin@nat@width>\linewidth\linewidth\else\Gin@nat@width\fi}
\def\maxheight{\ifdim\Gin@nat@height>\textheight\textheight\else\Gin@nat@height\fi}
\makeatother
% Scale images if necessary, so that they will not overflow the page
% margins by default, and it is still possible to overwrite the defaults
% using explicit options in \includegraphics[width, height, ...]{}
\setkeys{Gin}{width=\maxwidth,height=\maxheight,keepaspectratio}
% Set default figure placement to htbp
\makeatletter
\def\fps@figure{htbp}
\makeatother
\setlength{\emergencystretch}{3em} % prevent overfull lines
\providecommand{\tightlist}{%
  \setlength{\itemsep}{0pt}\setlength{\parskip}{0pt}}
\setcounter{secnumdepth}{-\maxdimen} % remove section numbering
\ifLuaTeX
  \usepackage{selnolig}  % disable illegal ligatures
\fi
\usepackage{bookmark}
\IfFileExists{xurl.sty}{\usepackage{xurl}}{} % add URL line breaks if available
\urlstyle{same}
\hypersetup{
  pdftitle={Fichamento do Texto: Considerações teórico-metodológicas sobre as origens e a inserção do sistema de informação geográfica na geografia},
  pdfauthor={Lucas Hernandes da Costa Porto, NUSP: 11918140},
  hidelinks,
  pdfcreator={LaTeX via pandoc}}

\title{Fichamento do Texto: Considerações teórico-metodológicas sobre as
origens e a inserção do sistema de informação geográfica na geografia}
\author{Lucas Hernandes da Costa Porto, NUSP: 11918140}
\date{}

\begin{document}
\maketitle

O texto aborda a evolução dos Sistemas de Informação Geográfica (SIG),
destacando seu crescimento tanto no Brasil quanto em outras partes do
mundo, como parte da transição para uma sociedade informacional. Esse
crescimento é impulsionado por fatores como a expansão dos investimentos
em informática, o desenvolvimento de microprocessadores mais rápidos, e
a redução dos preços de microcomputadores e periféricos, levando ao
aumento da comunidade de usuários de informações espaciais. Esse cenário
impulsionou a discussão sobre conceitos e paradigmas da análise
espacial, essenciais para a existência dos SIGs.

Os SIGs têm se adaptado a diversas aplicações na geografia, como o
mapeamento geotécnico e a análise de bacias hidrográficas, em grande
parte graças à integração com sistemas CAD e tecnologias de
sensoriamento remoto. Entretanto, o texto ressalta uma distinção
importante entre a simples acumulação de dados e a sabedoria geográfica,
que envolve conhecimento e contexto, essenciais para o uso eficaz dos
SIGs. O autor também destaca a importância de buscar as origens e os
fundamentos do pensamento espacial na geografia para orientar as
pesquisas contemporâneas que utilizam essas ferramentas geotecnológicas.

A análise espacial, como explorada no texto, não surgiu com o advento
dos computadores, mas sim da tradição geométrica e espacial da geografia
anglo-saxônica dos meados do século XX. Aqui, são diferenciadas as
escolas corológica e espacial como abordagens antagônicas, mas
complementares, dentro da geografia. A escola corológica se concentra na
descrição de regiões específicas e suas características, enquanto a
escola espacial enfatiza o arranjo geométrico dos fenômenos e a análise
de padrões espaciais. A importância da perspectiva espacial é sublinhada
pela citação de Coffey (1981), que destaca como a localização absoluta
ou relativa pode influenciar a intensidade das propriedades não
espaciais.

Além disso, o texto explora como a análise espacial contemporânea se
tornou sinônimo de um conjunto de técnicas de manipulação de dados
espaciais, fortemente influenciado pela estatística. A escola espacial,
ao focar na distribuição relativa de objetos no espaço e nos padrões
geométricos, contrasta com a escola corológica. Sack (1974) é citado
para reforçar que o estudo da distribuição espacial dos fenômenos é
central para a escola espacial, destacando o papel crucial do mapa na
geografia e nos SIGs.

A dualidade entre múltiplas características em um mesmo local e a
distribuição dessas características em vários locais dentro da matriz
geográfica de Berry também é abordada. Berry (1964) sugere que a
complexidade do espaço geográfico é compreendida através da expansão das
dimensões das séries de características e locais, formando uma matriz
geográfica que pode ser representada por mapas temáticos. Ele propôs que
a superposição de mapas temáticos em camadas poderia representar a
totalidade do espaço geográfico, um conceito fundamental para o
desenvolvimento dos SIGs.

O texto ainda discute os modelos e paradigmas da informação geográfica,
como o modelo de objetos exatos e o modelo de campos contínuos, ambos
essenciais para a representação espacial em SIG. A evolução da geografia
espacial e da análise quantitativa para integrar as metodologias dos
SIGs é destacada, com Berry sendo citado como um dos pioneiros na
proposição de técnicas que contribuíram para o desenvolvimento das
funções de análise espacial nos SIGs. Em última análise, conclui-se que
o SIG não é um paradigma da informática, mas uma consequência natural da
evolução das teorias da análise espacial, que hoje são implementadas
através da codificação vetor-raster e da representação orientada a
objetos, permitindo a modelagem e a análise eficiente do espaço
geográfico.

\textbf{Referência:} Ferreira, M. C. (2006). Considerações
teórico-metodológicas sobre as origens e a inserção do sistema de
informação geográfica na geografia. VITTE, Antonio Carlos. Contribuições
à história e à epistemologia da geografia. Rio de Janeiro: Bertrand
Brasil, 101-125.

\end{document}