\documentclass{beamer}

% Escolha um tema
\usetheme{Madrid} % Exemplos: Madrid, Copenhagen, Berlin, etc.
% \usecolortheme{seagull} % Escolha uma paleta de cores: seagull, dolphin, etc.

% Configurações opcionais
\setbeamertemplate{navigation symbols}{} % Remove símbolos de navegação
\setbeamertemplate{footline}[frame number] % Mostra o número do slide no rodapé

% Informações do documento
\title{Utilização de áreas de influência e Buffers na avaliação de acessibilidade hospitalar em São Paulo - SP}
\author{Ana Carolina da Silva, Laura Gonçalves Ferreira,  Lucas Hernandes da Costa Porto, Lucas Lima Gomes da Silva}
\institute{Universidade de São Paulo}
\date{5 de Dezembro de 2024} % Ou uma data personalizada

\begin{document}

% Slide inicial (título)
\begin{frame}
    \titlepage
\end{frame}


\section{Introdução}
\begin{frame}{Introdução}
    \begin{columns}
        % Coluna 1: Texto
        \begin{column}{0.6\textwidth}
            \begin{itemize}
                \item Hospitais: 241 hospitais dos quais \textbf{53 hospitais pertencem a rede pública}.
                \item \textbf{\textcolor{red}{2,48}} leitos totais por mil habitantes. \textbf{\textcolor{red}{0,92}} leitos públicos por mil habitantes. A OMS recomenda \textbf{\textcolor{blue}{3 leitos totais}} por mil habitantes.
                \item Região central: 14,8 leitos totais/mil. 1,04 leitos públicos/mil.
                \item Zona Leste: 1,5 leitos totais/mil. 0,6 leitos públicos/mil.
            \end{itemize}
        \end{column}

        % Coluna 2: Figura
        \begin{column}{0.4\textwidth}
            \centering
            \includegraphics[width=\textwidth]{plots/mapa_hospitais.png} 
        \end{column}
    \end{columns}
\end{frame}


\section{Áreas de Influência}

\begin{frame}{Áreas de Influência}
    \begin{block}{Poligonos de Voronói}
      \begin{itemize}
        \item Dividem o espaço com base na proximidade a pontos específicos.
        \item Representam a área de influência teórica de serviços como hospitais.
        \item Úteis para entender a distribuição espacial de serviços.
      \end{itemize}
    \end{block}

    \begin{block}{Buffers}
      \begin{itemize}
        \item Zonas de 2Km ao redor dos hospitais.
        \item Simulam áreas de alcance prático, considerando deslocamentos reais.
        \item Complementam os Voronoi com uma abordagem mais prática.
      \end{itemize}    \end{block}
\end{frame}

\begin{frame}{Áreas de Influência}
    \begin{columns}
        % Coluna 1: Texto
        \begin{column}{0.5\textwidth}
            \centering
            \includegraphics[width=\textwidth]{plots/acessibilidade_teorica.png} 
        \end{column}

        % Coluna 2: Figura
        \begin{column}{0.5\textwidth}
            \centering
            \includegraphics[width=\textwidth]{plots/acessibilidade_real.png} 
        \end{column}
    \end{columns}
\end{frame}

\begin{frame}{Áreas de Influência}
    \begin{columns}
        % Coluna 1: Texto
        \begin{column}{0.5\textwidth}
            \includegraphics[width=\textwidth]{plots/acessibilidade_teorica_leitos_mpcorr.png} 
        \end{column}

        % Coluna 2: Figura
        \begin{column}{0.5\textwidth}
            \includegraphics[width=\textwidth]{plots/acessibilidade_real_leitos_mpcorr.png} 
        \end{column}
    \end{columns}
\end{frame}

\begin{frame}{Áreas de Influência}
    \begin{columns}
        % Coluna 1: Texto
        \begin{column}{0.5\textwidth}
            \includegraphics[width=\textwidth]{plots/acessibilidade_real_zl.png} 
        \end{column}

        % Coluna 2: Figura
        \begin{column}{0.5\textwidth}
            \includegraphics[width=\textwidth]{plots/acessibilidade_real_renda.png} 
        \end{column}
    \end{columns}
\end{frame}

\begin{frame}{Áreas de Influência}
    \begin{block}{Acessibilidade teórica}
      \begin{itemize}
        \item Os polígonos de Voronoi abrangem toda a área urbana, mas exibem uma variação considerável em suas dimensões
        \item Hospitais localizados em regiões centrais geram polígonos menores, reflexo da maior densidade de unidades hospitalares na área.
        \item Por outro lado, hospitais situados em zonas periféricas geram áreas de influência mais amplas, o que indica uma menor densidade de serviços nesses locais
      \end{itemize}
    \end{block}
\end{frame}

\begin{frame}{Áreas de Influência}
    \begin{block}{Acessibilidade real}
      \begin{itemize}
        \item Os buffers indicaram que aproximadamente 64\% da população está situada dentro do raio de 2 km
        \item Regiões de até 2 salários minimos não possuem cobertura hospitar abrangente
      \end{itemize}   
    \end{block}
\end{frame}

\section{Conclusão}
\begin{frame}{Conclusão}
    \begin{itemize}
        \item Os resultados apontam para a necessidade de redistribuição estratégica dos serviços hospitalares, especialmente em regiões periféricas, onde as lacunas de acessibilidade são mais evidentes.
        \item Com base nos achados, recomenda-se a instalação de novos hospitais em áreas atualmente subatendidas, bem como a melhoria no transporte público para facilitar o deslocamento da população. O uso combinado dessas ferramentas oferece uma base sólida para embasar políticas públicas e promover maior equidade no acesso aos serviços de saúde.
    \end{itemize}
\end{frame}

% Slide final
\begin{frame}
    \centering
    \Huge Obrigado!
\end{frame}

\end{document}
