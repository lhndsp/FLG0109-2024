% Options for packages loaded elsewhere
\PassOptionsToPackage{unicode}{hyperref}
\PassOptionsToPackage{hyphens}{url}
%
\documentclass[
]{article}
\usepackage{amsmath,amssymb}
\usepackage{iftex}
\ifPDFTeX
  \usepackage[T1]{fontenc}
  \usepackage[utf8]{inputenc}
  \usepackage{textcomp} % provide euro and other symbols
\else % if luatex or xetex
  \usepackage{unicode-math} % this also loads fontspec
  \defaultfontfeatures{Scale=MatchLowercase}
  \defaultfontfeatures[\rmfamily]{Ligatures=TeX,Scale=1}
\fi
\usepackage{lmodern}
\ifPDFTeX\else
  % xetex/luatex font selection
\fi
% Use upquote if available, for straight quotes in verbatim environments
\IfFileExists{upquote.sty}{\usepackage{upquote}}{}
\IfFileExists{microtype.sty}{% use microtype if available
  \usepackage[]{microtype}
  \UseMicrotypeSet[protrusion]{basicmath} % disable protrusion for tt fonts
}{}
\makeatletter
\@ifundefined{KOMAClassName}{% if non-KOMA class
  \IfFileExists{parskip.sty}{%
    \usepackage{parskip}
  }{% else
    \setlength{\parindent}{0pt}
    \setlength{\parskip}{6pt plus 2pt minus 1pt}}
}{% if KOMA class
  \KOMAoptions{parskip=half}}
\makeatother
\usepackage{xcolor}
\usepackage[margin=2cm]{geometry}
\usepackage{graphicx}
\makeatletter
\def\maxwidth{\ifdim\Gin@nat@width>\linewidth\linewidth\else\Gin@nat@width\fi}
\def\maxheight{\ifdim\Gin@nat@height>\textheight\textheight\else\Gin@nat@height\fi}
\makeatother
% Scale images if necessary, so that they will not overflow the page
% margins by default, and it is still possible to overwrite the defaults
% using explicit options in \includegraphics[width, height, ...]{}
\setkeys{Gin}{width=\maxwidth,height=\maxheight,keepaspectratio}
% Set default figure placement to htbp
\makeatletter
\def\fps@figure{htbp}
\makeatother
\setlength{\emergencystretch}{3em} % prevent overfull lines
\providecommand{\tightlist}{%
  \setlength{\itemsep}{0pt}\setlength{\parskip}{0pt}}
\setcounter{secnumdepth}{-\maxdimen} % remove section numbering
\ifLuaTeX
  \usepackage{selnolig}  % disable illegal ligatures
\fi
\usepackage{bookmark}
\IfFileExists{xurl.sty}{\usepackage{xurl}}{} % add URL line breaks if available
\urlstyle{same}
\hypersetup{
  pdftitle={Fichamento do Artigo: Os novos enfoques da Geografia com o apoio das Tecnologias da Informação Geográfica},
  pdfauthor={Lucas Hernandes da Costa Porto, NUSP: 11918140},
  hidelinks,
  pdfcreator={LaTeX via pandoc}}

\title{Fichamento do Artigo: Os novos enfoques da Geografia com o apoio das Tecnologias da Informação Geográfica}
\author{Lucas Hernandes da Costa Porto, NUSP: 11918140}
\date{}

\begin{document}
\maketitle

A geografia, ao longo do século XX e início do XXI, passou por transformações significativas em seus paradigmas, que moldaram as formas de entender e representar o espaço geográfico. Esses paradigmas — Quantitativo, Regional, Humanista e outros — não se excluem, mas frequentemente coexistem e se complementam, enriquecendo o campo com abordagens diversas. O avanço das Tecnologias da Informação Geográfica (TIG) trouxe novos desafios e oportunidades, transformando o debate contemporâneo na área.

\section{Principais Paradigmas e Suas Abordagens}

\subsection{Geografia Quantitativa}

\begin{itemize}
  \item \textbf{Características:} Surgiu com força nos anos 1950 e 1960, influenciada pelo positivismo lógico. Essa abordagem utiliza métodos matemáticos, estatísticos e ferramentas computacionais para analisar padrões espaciais. Busca explicar fenômenos por meio de regularidades, como modelos de localização ou análise de fluxos.
  \item \textbf{Marco histórico:} O movimento foi impulsionado por Peter Haggett e Brian Berry, que incorporaram técnicas como a análise de regressão e a teoria dos sistemas.
  \item \textbf{Atualidade:} Ressurgiu com o advento das Geotecnologias, como SIG (Sistemas de Informação Geográfica), permitindo modelagens mais complexas e precisas.
\end{itemize}

\subsection{Geografia Regional}

\begin{itemize}
  \item \textbf{Características:} Predominante no início do século XX, essa abordagem analisa as regiões como unidades únicas, integrando fatores naturais, econômicos e culturais. Tem uma perspectiva descritiva e integradora.
  \item \textbf{Marco histórico:} Foi consolidada por Vidal de la Blache, com a geografia possibilista, que enfatizava a relação adaptativa entre o ser humano e o ambiente.
  \item \textbf{Atualidade:} Continua sendo uma base para estudos que necessitam de entendimento detalhado e holístico de territórios específicos.
\end{itemize}

\subsection{Geografia Humanista}

\begin{itemize}
  \item \textbf{Características:} Emergindo nos anos 1970 como reação ao determinismo da Geografia Quantitativa, enfatiza as percepções, os significados e as experiências humanas em relação ao espaço.
  \item \textbf{Marco histórico:} Influenciada pelos estudos culturais e pela fenomenologia, com autores como Yi-Fu Tuan, que exploraram os conceitos de lugar e espaço vivido.
  \item \textbf{Atualidade:} Ganhou novas perspectivas com estudos sobre mapas mentais e representações subjetivas do espaço.
\end{itemize}

\subsection{Geografia Racionalista}

\begin{itemize}
  \item \textbf{Características:} Baseia-se no pensamento lógico e cartesiano, enfatizando o planejamento e a organização funcional do espaço. Embora menos destacado como paradigma isolado, integra-se a outras correntes na busca por soluções práticas.
\end{itemize}

Os paradigmas geográficos não eliminam uns aos outros, mas coexistem e evoluem juntos. Isso reflete a dinâmica do conhecimento científico, como descrito por Thomas Kuhn (1970), em que novos paradigmas deslocam os anteriores sem necessariamente extingui-los. Assim:

\begin{itemize}
  \item Geografia Quantitativa e Humanista: Embora pareçam opostas, suas ferramentas frequentemente se complementam. Modelos quantitativos podem ser enriquecidos com insights qualitativos, como mapas participativos ou análise de percepção.
  \item Geografia Regional e Quantitativa: Estudos regionais detalhados podem servir de base para modelagens quantitativas mais robustas.
  \item Geotecnologias: Estão no centro dessa integração, proporcionando meios para unir análises matemáticas e qualitativas. Exemplos incluem SIGs colaborativos e análises que mesclam dados objetivos e subjetivos.
\end{itemize}

\section{Marcos Contemporâneos e Desafios}

Com a ascensão das Tecnologias da Informação Geográfica, como drones, SIGs e big data, a geografia enfrenta uma nova mudança paradigmática. Essa evolução é marcada por buscar consolidar fundamentos teóricos e ontológicos para integrar a análise espacial, computação e modelagem temática. Os desafios consistem na necessidade de equilibrar o avanço tecnológico com a compreensão humanista, garantindo que o espaço geográfico seja visto tanto como uma abstração matemática quanto como um lugar vivido e experimentado.


\section{Conclusão}

A geografia contemporânea é uma disciplina plural, que se enriquece pela coexistência de paradigmas. O avanço das Geotecnologias tem impulsionado a interação entre abordagens quantitativas e qualitativas, permitindo análises mais profundas e integradas do espaço. Essa pluralidade fortalece a disciplina, tornando-a central para enfrentar os desafios do século XXI, como planejamento urbano sustentável, mudanças climáticas e desigualdades territoriais.


\end{document}